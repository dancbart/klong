\documentclass{article}
\usepackage{amsmath}

\begin{document}

\begin{equation}
    E^2 = p^2 c^2 + m^2 c^4
\end{equation}

\textbf{Reaction:}
\[ K_{\text{long}} + p \rightarrow K^+ + n \]

\textbf{Conservation of Energy:}
\begin{equation}
    E_i + m_p c^2 = E_f + E_n
\end{equation}

\textbf{Conservation of Momentum:}
\begin{equation}
    \vec{p}_i = \vec{p}_f + \vec{p}_n
\end{equation}
    
\begin{equation}
    \cos(\theta) = \frac{k_{\text{Long}_p} + k_{\text{Plus}_p} + n_m^2 - \sqrt{k_{\text{Plus}_p}^2 + k_{p_m}^2} - n_e + \sqrt{k_{p_m}^2 + k_{\text{Plus}_p}^2}}{2 \cdot k_{\text{Long}_p} \cdot k_{\text{Plus}_p}}
    \end{equation}
    
    where:
    \begin{align*}
    k_{\text{Long}_p} & \text{ is the Longitudinal Momentum} \\
    k_{\text{Plus}_p} & \text{ is the Momentum of K+} \\
    n_m & \text{ is the Neutron Mass (recoil)} \\
    n_e & \text{ is the Neutron Energy (recoil)} \\
    k_{p_m} & \text{ is the Mass of K+}
    \end{align*}
    
    This formula describes the calculation of $\cos(\theta)$ in terms of the given variables.

\end{equation}


\end{document}
